% CVPR 2022 Paper Template
% based on the CVPR template provided by Ming-Ming Cheng (https://github.com/MCG-NKU/CVPR_Template)
% modified and extended by Stefan Roth (stefan.roth@NOSPAMtu-darmstadt.de)

\documentclass[10pt,twocolumn,letterpaper]{article}

%%%%%%%%% PAPER TYPE  - PLEASE UPDATE FOR FINAL VERSION
%\usepackage[review]{cvpr}      % To produce the REVIEW version
\usepackage{cvpr}              % To produce the CAMERA-READY version
%\usepackage[pagenumbers]{cvpr} % To force page numbers, e.g. for an arXiv version

% Include other packages here, before hyperref.
\usepackage{graphicx}
\usepackage{amsmath}
\usepackage{amssymb}
\usepackage{booktabs}


% It is strongly recommended to use hyperref, especially for the review version.
% hyperref with option pagebackref eases the reviewers' job.
% Please disable hyperref *only* if you encounter grave issues, e.g. with the
% file validation for the camera-ready version.
%
% If you comment hyperref and then uncomment it, you should delete
% ReviewTempalte.aux before re-running LaTeX.
% (Or just hit 'q' on the first LaTeX run, let it finish, and you
%  should be clear).
\usepackage[pagebackref,breaklinks,colorlinks]{hyperref}


% Support for easy cross-referencing
\usepackage[capitalize]{cleveref}
\crefname{section}{Sec.}{Secs.}
\Crefname{section}{Section}{Sections}
\Crefname{table}{Table}{Tables}
\crefname{table}{Tab.}{Tabs.}


%%%%%%%%% PAPER ID  - PLEASE UPDATE
\def\cvprPaperID{*****} % *** Enter the CVPR Paper ID here
\def\confName{CVPR}
\def\confYear{2022}


\begin{document}

%%%%%%%%% TITLE - PLEASE UPDATE
\title{Open-Set Domain Adaptation through Self-Supervision}

\author{Andrea Protopapa, Matteo Quarta, Giuseppe Ruggeri, Alessandro Versace\\
Politecnico di Torino\\
Italy\\
{\tt\small \{s286302,s292477,s292459,s292477\}@studenti.polito.it}
% For a paper whose authors are all at the same institution,
% omit the following lines up until the closing ``}''.
% Additional authors and addresses can be added with ``\and'',
% just like the second author.
% To save space, use either the email address or home page, not both
}
\maketitle

%%%%%%%%% ABSTRACT
\begin{abstract}
   
\end{abstract}

%%%%%%%%% BODY TEXT
\section{Introduction}
\label{sec:intro}

Start write here

%-------------------------------------------------------------------------
\subsection{Notes}
%---------- LANGUAGE ----------
All manuscripts must be in English.
%---------- PAPER LENGTH ----------
Papers, excluding the references section, must be no longer than eight pages in length.
The references section will not be included in the page count, and there is no limit on the length of the references section.
%---------- MATH ----------
Please number all of your sections and displayed equations as in these examples:
\begin{equation}
  E = m\cdot c^2
  \label{eq:important}
\end{equation}
and
\begin{equation}
  v = a\cdot t.
  \label{eq:also-important}
\end{equation}
All authors will benefit from reading Mermin's description of how to write mathematics:
\url{http://www.pamitc.org/documents/mermin.pdf}.
%---------- BLIND VIEW ----------
Blind review means that you do not use the words ``my'' or ``our'' when citing previous work.

Saying ``this builds on the work of Lucy Smith [1]'' does not say that you are Lucy Smith;
it says that you are building on her work.
If you are Smith and Jones, do not say ``as we show in [7]'', say ``as Smith and Jones show in [7]'' and at the end of the paper, include reference 7 as you would any other cited work.

An example of a bad paper just asking to be rejected:
\begin{quote}
\begin{center}
    An analysis of the frobnicatable foo filter.
\end{center}

   In this paper we present a performance analysis of our previous paper [1], and show it to be inferior to all previously known methods.
   Why the previous paper was accepted without this analysis is beyond me.

   [1] Removed for blind review
\end{quote}


An example of an acceptable paper:
\begin{quote}
\begin{center}
     An analysis of the frobnicatable foo filter.
\end{center}

   In this paper we present a performance analysis of the  paper of Smith \etal [1], and show it to be inferior to all previously known methods.
   Why the previous paper was accepted without this analysis is beyond me.

   [1] Smith, L and Jones, C. ``The frobnicatable foo filter, a fundamental contribution to human knowledge''. Nature 381(12), 1-213.
\end{quote}

Finally, you may feel you need to tell the reader that more details can be found elsewhere, and refer them to a technical report.
For conference submissions, the paper must stand on its own, and not {\em require} the reviewer to go to a tech report for further details.
Thus, you may say in the body of the paper ``further details may be found in~\cite{Authors14b}''.
Then submit the tech report as supplemental material.
Again, you may not assume the reviewers will read this material.
%---------- CAPTION EXAMPLE ----------
\begin{figure}[t]
  \centering
  \fbox{\rule{0pt}{2in} \rule{0.9\linewidth}{0pt}}
   %\includegraphics[width=0.8\linewidth]{egfigure.eps}

   \caption{Example of caption.
   It is set in Roman so that mathematics (always set in Roman: $B \sin A = A \sin B$) may be included without an ugly clash.}
   \label{fig:onecol}
\end{figure}
%---------- MISCELLANEOUS: E.G., ET AL. ----------
The space after \eg, meaning ``for example'', should not be a sentence-ending space.
So \eg is correct, {\em e.g.} is not.
The provided \verb'\eg' macro takes care of this.

When citing a multi-author paper, you may save space by using ``et alia'', shortened to ``\etal'' (not ``{\em et.\ al.}'' as ``{\em et}'' is a complete word).
If you use the \verb'\etal' macro provided, then you need not worry about double periods when used at the end of a sentence as in Alpher \etal.
However, use it only when there are three or more authors.
%---------- SUB-FIGURES EXAMPLE ----------
\begin{figure*}
  \centering
  \begin{subfigure}{0.68\linewidth}
    \fbox{\rule{0pt}{2in} \rule{.9\linewidth}{0pt}}
    \caption{An example of a subfigure.}
    \label{fig:short-a}
  \end{subfigure}
  \hfill
  \begin{subfigure}{0.28\linewidth}
    \fbox{\rule{0pt}{2in} \rule{.9\linewidth}{0pt}}
    \caption{Another example of a subfigure.}
    \label{fig:short-b}
  \end{subfigure}
  \caption{Example of a short caption, which should be centered.}
  \label{fig:short}
\end{figure*}
%---------- FORMATTING ----------
All text must be in a two-column format.
%---------- FOOTNOTES ----------
Please use footnotes\footnote{This is what a footnote looks like.
It often distracts the reader from the main flow of the argument.} sparingly.
Indeed, try to avoid footnotes altogether and include necessary peripheral observations in the text (within parentheses, if you prefer, as in this sentence).
If you wish to use a footnote, place it at the bottom of the column on the page on which it is referenced.
Use Times 8-point type, single-spaced.
%---------- CROSS-REFERENCES ----------
For the benefit of author(s) and readers, please use the
{\small\begin{verbatim}
  \cref{...}
\end{verbatim}}  command for cross-referencing to figures, tables, equations, or sections.
This will automatically insert the appropriate label alongside the cross-reference as in this example:
\begin{quotation}
  To see how our method outperforms previous work, please see \cref{fig:onecol} and \cref{tab:example}.
  It is also possible to refer to multiple targets as once, \eg~to \cref{fig:onecol,fig:short-a}.
  You may also return to \cref{sec:formatting} or look at \cref{eq:also-important}.
\end{quotation}
If you do not wish to abbreviate the label, for example at the beginning of the sentence, you can use the
{\small\begin{verbatim}
  \Cref{...}
\end{verbatim}}
command. Here is an example:
\begin{quotation}
  \Cref{fig:onecol} is also quite important.
\end{quotation}
%---------- CAPTION REFERENCES ----------
% Update the cvpr.cls to do the following automatically.
% For this citation style, keep multiple citations in numerical (not
% chronological) order, so prefer \cite{Alpher03,Alpher02,Authors14} to
% \cite{Alpher02,Alpher03,Authors14}.
List and number all bibliographical references at the end of your paper.
When referenced in the text, enclose the citation number in square brackets, for
example~\cite{Authors14}.
Where appropriate, include page numbers and the name(s) of editors of referenced books.
When you cite multiple papers at once, please make sure that you cite them in numerical order like this \cite{Alpher02,Alpher03,Alpher05,Authors14b,Authors14}.
If you use the template as advised, this will be taken care of automatically.
%---------- TABLE EXAMPLE ----------
\begin{table}
  \centering
  \begin{tabular}{@{}lc@{}}
    \toprule
    Method & Frobnability \\
    \midrule
    Theirs & Frumpy \\
    Yours & Frobbly \\
    Ours & Makes one's heart Frob\\
    \bottomrule
  \end{tabular}
  \caption{Results.   Ours is better.}
  \label{tab:example}
\end{table}
%---------- CENTERING GRAPHICS ----------
All graphics should be centered.
In \LaTeX, avoid using the \texttt{center} environment for this purpose, as this adds potentially unwanted whitespace.
Instead use
{\small\begin{verbatim}
  \centering
\end{verbatim}}
at the beginning of your figure.

When placing figures in \LaTeX, it's almost always best to use \verb+\includegraphics+, and to specify the figure width as a multiple of the line width as in the example below
{\small\begin{verbatim}
   \usepackage{graphicx} ...
   \includegraphics[width=0.8\linewidth]
                   {myfile.pdf}
\end{verbatim}
}
%---------- COLOR ----------
If you use color in your plots, please keep in mind that a significant subset of reviewers and readers may have a color vision deficiency; red-green blindness is the most frequent kind.
Hence avoid relying only on color as the discriminative feature in plots (such as red \vs green lines), but add a second discriminative feature to ease disambiguation.



%------------------------------------------------------------------------
\section{Related Work}
\label{sec:relatedwork}




%------------------------------------------------------------------------
\section{Method - Write name of proposal approach}
\label{sec:method}
%------------------------------------------------------------------------
\subsection{Subsection 1}
%------------------------------------------------------------------------
\subsection{Subsection 2}

%------------------------------------------------------------------------
\section{Experiments}
\label{sec:experiments}
%------------------------------------------------------------------------
\subsection{Subsection 1 - Setup}
%------------------------------------------------------------------------
\subsection{Subsection 2 - Implementation Details}
%------------------------------------------------------------------------
\subsection{Subsection 3 - Results}

%------------------------------------------------------------------------
\section{Conclusions (and Future Work)}
\label{sec:conclusions}

%%%%%%%%% APPENDIX - TAKEN FROM ANOTHER TEMPLATE!!
\appendix
\section{Appendices}
\label{sec:appendix}
Appendices are material that can be read, and include lemmas, formulas, proofs, and tables that are not critical to the reading and understanding of the paper. 
Appendices should be \textbf{uploaded as supplementary material} when submitting the paper for review.
Upon acceptance, the appendices come after the references, as shown here.

\paragraph{\LaTeX-specific details:}
Use {\small\verb|\appendix|} before any appendix section to switch the section numbering over to letters.


\section{Supplemental Material}
\label{sec:supplemental}
Submissions may include non-readable supplementary material used in the work and described in the paper.
Any accompanying software and/or data should include licenses and documentation of research review as appropriate.
Supplementary material may report preprocessing decisions, model parameters, and other details necessary for the replication of the experiments reported in the paper.
Seemingly small preprocessing decisions can sometimes make a large difference in performance, so it is crucial to record such decisions to precisely characterize state-of-the-art methods. 

Nonetheless, supplementary material should be supplementary (rather than central) to the paper.
\textbf{Submissions that misuse the supplementary material may be rejected without review.}
Supplementary material may include explanations or details of proofs or derivations that do not fit into the paper, lists of
features or feature templates, sample inputs and outputs for a system, pseudo-code or source code, and data.
(Source code and data should be separate uploads, rather than part of the paper).

The paper should not rely on the supplementary material: while the paper may refer to and cite the supplementary material and the supplementary material will be available to the reviewers, they will not be asked to review the supplementary material.

%%%%%%%%% REFERENCES
{\small
\bibliographystyle{ieee_fullname}
\bibliography{egbib}
}

\end{document}
